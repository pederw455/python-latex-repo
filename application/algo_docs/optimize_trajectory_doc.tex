\section{Optimization of Spacecraft Trajectory}

\subsection{Introduction}

In this section, we present an algorithm designed to optimize a spacecraft's trajectory by maximizing fuel efficiency and utilizing gravitational assists. The algorithm calculates the optimal time and thrust parameters to minimize the cost function, which incorporates several penalties and efficiencies.

\subsection{Variables}

\begin{itemize}
    \item $m_{\text{fuel}}$: Total mass of fuel onboard in kilograms (kg).
    \item $T_{\text{initial}}$: Initial thrust in Newtons (N).
    \item $d_{\text{closest}}$: Closest approach distance to the celestial body in meters (m).
    \item $t$: Time for which the thrust is applied in seconds (s).
    \item $T$: Thrust applied in Newtons (N).
    \item $\Delta v$: Change in velocity (delta-v) in meters per second (m/s).
    \item $E_{\text{grav}}$: Effect of gravitational assist.
    \item $d_{\text{traveled}}$: Distance traveled by the spacecraft in meters (m).
    \item $m_{\text{used}}$: Fuel mass used in kilograms (kg).
    \item $P_{\text{time}}$: Penalty for the elapsed time.
    \item $P_{\text{efficiency}}$: Penalty based on efficiency related to the closest approach.
    \item $C$: Total cost function value.
\end{itemize}

\subsection{Formulas}

\begin{align*}
\Delta v &= \frac{T \cdot t}{m_{\text{fuel}}} \\
E_{\text{grav}} &= 
\begin{cases} 
\ln(d_{\text{closest}}) & \text{if } d_{\text{closest}} > 1 \\
0 & \text{otherwise}
\end{cases} \\
d_{\text{traveled}} &= \frac{\Delta v \cdot t \cdot E_{\text{grav}}}{2} \\
m_{\text{used}} &= m_{\text{fuel}} - \frac{T \cdot t}{\Delta v} \\
P_{\text{time}} &= 0.05 \cdot t^{1.2} \\
P_{\text{efficiency}} &= 0.01 \cdot (d_{\text{closest}} - 7 \times 10^6)^2 \\
C &= m_{\text{used}} + P_{\text{time}} + P_{\text{efficiency}}
\end{align*}