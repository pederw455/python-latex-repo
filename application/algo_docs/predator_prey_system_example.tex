\subsection{Example}

Consider a predator-prey system with the following parameters and initial conditions:

\begin{align*}
    \alpha &= 0.1 \quad \text{(prey growth rate)} \\
    \beta &= 0.02 \quad \text{(rate at which predators consume prey)} \\
    \delta &= 0.01 \quad \text{(rate at which prey consumption converts into predator growth)} \\
    \gamma &= 0.1 \quad \text{(predator death rate)} \\
    p_{\text{prey, initial}} &= 40 \quad \text{(initial population of prey)} \\
    p_{\text{predator, initial}} &= 9 \quad \text{(initial population of predators)}
\end{align*}

The time variable $t$ represents the temporal progression of the system:

\begin{align*}
    t &= [0, 200] \quad \text{(time points for simulation at intervals)}
\end{align*}

These equations describe how the populations of prey and predators change over time based on the initial conditions and parameters defined. The solution of these differential equations provides insights into the dynamics of the predator and prey populations within the specified time frame.